% !Mode:: "TeX:UTF-8"
%hithesis-alpha 用来测试添加开题/中期报告功能,非正式
%\documentclass[newtxmath=true,newgeometry=two,capcenterlast=true,subcapcenterlast=true,openright=true,absupper=true,fontset=windowsnew,type=doctor]{hithesis}
%\newcommand{\customheader}{自定义页眉}	
%在这里自定义页眉文字(此命令需在最前),并在documentclass中设置headersetting=customizeheader	_modified by hithesis-alpha
\documentclass[newtxmath=true,newgeometry=two,capcenterlast=true,subcapcenterlast=true,openright=false,absupper=true,stage=kaiti,type=master]{hithesis}

%%开题/中期等报告无右翻页问题,需设置openright=false,大论文时需参照原hithesis说明自行设置(一般博士论文在双面打印才需要设置右翻页) 
%%如果改回大论文模板编译出现makecover错误,记得在下面修改回原模板的封面文件哦\input{front/cover}		_modified by hithesis-alpha:
% 此处选项中不要有空格
%%%%%%%%%%%%%%%%%%%%%%%%%%%%%%%%%%%%%%%%%%%%%%%%%%%%%%%%%%%%%%%%%%%%%%%%%%%%%%%%
% 必填选项
% type=doctor|master|bachelor
%%%%%%%%%%%%%%%%%%%%%%%%%%%%%%%%%%%%%%%%%%%%%%%%%%%%%%%%%%%%%%%%%%%%%%%%%%%%%%%%
% 选填选项(选填选项的缺省值已经尽可能满足了大多数需求,除非明确知道自己有什么
% 需求)
% glue=true|false
% 	含义:由于我工规范中要求字体行距在一个闭区间内,这个选项为true表示tex自
% 	动选择,为false表示区间内一个最接近版心要求行数的要求的默认值,缺省值为
% 	false。
% tocfour=true|false
% 	含义:是否添加第四级目录,只对本科文科个别要求四级目录有效,缺省值为
% 	false
% fontset=siyuan|windowsnew|windowsold
% 	含义:注意这个选项视为了解决特殊问题而设置,比如用有些发行版本的linux排
% 	版时可能(大多数发行版不会)会遇到的字体无法载入的问题,或者字体载入之
% 	后出现无法复制的问题以及想要解决排版如 biang biang 面的 biang 这类中易
% 	宋体无法识别的汉字的问题。没有特殊的需要不推荐使用这个选项。
%
% 	如果是安装了 windowns 字体的 linux 系统,可以填写windowsnew(win vista
% 	以后 的字体)或 windowsold(vista 以前)或者想用思源宋体并且是已经安装
% 	了思源宋体的任何系统,填写siyuan选项。缺省值为空,自动识别系统并匹配字体
% 	。模板版中给出的思源字体定义文件定义的思源字体的版本是Adobe版,其他字体
% 	是windowsnew字体。
% tocblank=true|false
% 	含义:目录中第一章之前,是否加一行空白。缺省值为true。
% chapterhang=true|false
% 	含义:目录的章标题是否悬挂居中,规范中要求章标题少于15字,所以这个选项
% 	有无没什么用,除了特殊需求。缺省值为true。
% fulltime=true|false
% 	含义:是否全日制,缺省值为true。非全日制如同等学力等,要在cover中设置类
% 	型,封面中不同格式
% subtitle=true|false
% 	含义:论文题目是否含有副标题,缺省值为false,如果有要在cover中设置副标
% 	题内容,封面中显示。
% newgeometry=one|two
% 	含义:规范中的自相矛盾之处,版芯是否包含页眉页脚,旧方法是按照包含页眉
% 	页脚来设置。该选项是多选选项,如果没有这个选项,缺省值是旧模板的版芯设
% 	置方法,如果设置该选项one或two,分别对应两种页眉页码对应版芯线的相对位
% 	置。第一种是严格按照规范要求,难看。第二种微调了页眉页码位置,好一点。
% debug=true|false
% 	含义:是否显示版芯框和行号,用来调试。默认否。
% openright=true|false
% 	含义:博士论文是否要求章节首页必须在奇数页,此选项不在规范要求中,按个
% 	人喜好自行决定。 默认否。注意,窝工的默认情况是打印版博士论文要求右翻页
% 	,电子版要求非右翻页且无空白页。如果想DIY(或身不由己DIY)在什么地方右
% 	翻页,将这个选项设置为false,然后在目标位置添加`\cleardoublepage`命令即
% 	可。
% capcenterlast=true|false
% 	含义:图题、表题最后一行是否居中对齐(我工规范要求居中,但不要求居中对
% 	齐),此选项不在规范要求中,按个人喜好自行决定。默认否。
% subcapcenterlast=true|false
% 	含义:子图图题最后一行是否居中对齐(我工规范要求居中,但不要求居中对齐
% 	),此选项不在规范要求中,按个人喜好自行决定。默认否。
% absupper=true|false
%       含义:中文目录中的英文索引在中文目录中的大小写样式歧义,在规范中要求首
%       字母大写,在work样例中是全大写。该选项控制是否全大写。默认否。
% bsmainpagenumberline=true|false
%       含义:由于本科生论文官方模板的页码和页眉格式混乱,提供这个选项自定义设
%       置是否在正文中显示页码横线,默认否。
% bsfrontpagenumberline=true|false
%       含义:由于本科生论文官方模板的页码和页眉格式混乱,提供这个选项自定义设
%       置是否在前文中显示页码横线,默认否。
% bsheadrule=true|false
%       含义:由于本科生论文官方模板的页码和页眉格式混乱,提供这个选项自定义设
%       置是否显示页眉横线,默认显示。
% splitbibitem=true|false
%       含义:参考文献每一个条目内能不能断页,应广大刀客要求添加。默认否。
% newtxmath=true|false
%       含义:数学字体是否使用新罗马。默认是。

%% _modified by hithesis-alpha:
%%开题/中期报告需修改:
%		首先依然正常选择type=doctor/master/bachelor
%stage=kaiti/zhongqi
%		含义:开题报告或中期报告时选填,需放在type项前,去掉该项缺省为false本硕博毕业论文格式(去掉该项若makecover报错,请在下方切换封面页->\input{front/cover},)。
%headersetting=noheader/customizeheader
%		含义:自定义页眉选项(合并原noheader选项),删除该项default默认自动显示;
%				选择noheader:全局“没有页眉”,某些老师可能要求开题、中期报告没有页眉,如果在bachelor下使用,优先级高于bsheadrule;
%				选择customize:自定义页眉文字,需在文件开头填写自定义文字\newcommand{\customheader}{自定义页眉},可灵活用于综合考评、实验报告等其它场景。
%%%%%%%%%%%%%%%%%%%%%%%%%%%%%%%%%%%%%%%%%%%%%%%%%%%%%%%%%%%%%%%%%%%%%%%%%%%%%%%%
\fancyhf{}
\usepackage{hithesis}
\graphicspath{{figures/}}

%在下面添加自定义宏包、命令

%------------以下是自定义添加的宏包(下面为示例,可删除)------------------
\usepackage{leftidx} %左上标
\usepackage{multirow} %表格多行合并
\usepackage{slashed}	% 场论slash
\usepackage{cancel}

\usepackage{setspace} %用于调整多行公式间距, \begin{spacing}{1.6} \begin{equation}..\end{equation}\end{spacing}  !!!caution:使用宏包同时也会改变表格的行距需在每个表格\begin{table}后添加:\renewcommand\arraystretch{1.55}






%------------以下是自定义添加的命令(下面为示例,可删除)------------------

%重新定义section和subsection标号格式,去掉Chapter继承标号,因为开题等报告没有Chapter层级
%\renewcommand\thesection{\arabic {section}.}
%\renewcommand\thesubsection{\thesection \arabic {subsection}}

%-------罗马数字定义-------
\makeatletter
\newcommand{\rmnum}[1]{\romannumeral #1}
\newcommand{\Rmnum}[1]{\expandafter\@slowromancap\romannumeral #1@}
\makeatother


%---------正体字母定义------------
\newcommand{\ud}{\mathrm{d}}
\newcommand{\ue}{\mathrm{e}}
\newcommand{\ui}{\mathrm{i}}

%---------度°符号定义-------------
\def\degree{\ensuremath{{}^{\circ}}} 

%---------tabular行距------------
%\renewcommand{\arraystretch}{1.25}

	%插入自定义宏包、命令文件_t

\begin{document}

\frontmatter
%\setcounter{page}{0} %如果使用开题/中期报告时,目录页码不正确时使用,重置页码计数器
%\includepdfmerge{front/report_cover.pdf}	
%如果自动生成封面不满足需要,可以自行填写学校word模板封面,生成pdf文件,放入front文件夹,手动插入该pdf封面,并手动注释掉下面% !Mode:: "TeX:UTF-8"
%开题/中期等报告用的封面页
\hitsetup{
  %******************************
  % 注意:
  %   1. 配置里面不要出现空行
  %   2. 不需要的配置信息可以删除
  %******************************
  %
  %=====
  % 秘级
  %=====
  %
  %=========
  % 中文信息
  %=========
  ctitleone={局部多孔质气体静压局部多孔质气体静压}, %论文题目,第一行此处最多12个字,多出的字放在\ctitletwo
  ctitletwo={轴承关键技术的研究}, %选填项,论文题目第二行,第一行放不下的文字放在这里
  csubject={机械制造及其自动化},
  cclassid={1604104}, %班号,威海校区特有,仅选择威海校区时有效
  caffil={机电工程学院},
  cauthor={于冬梅},
  csupervisor={某某某教授},
  % ccosupervisor={某某某教授}, % 联合指导老师
  % 日期自动使用当前时间,若需指定按如下方式修改:
  cdate={超新星纪元},
  cstudentid={9527},
  %cstudenttype={同等学力人员}, %非全日制教育申请学位者
  %(同等学力人员)、(工程硕士)、(工商管理硕士)、
  %(高级管理人员工商管理硕士)、(公共管理硕士)、(中职教师)、(高校教师)等
  %
  %
  %配置
  %\renewcommand\thesection{\arabic {section}.}
  %\renewcommand\thesubsection{\thesection \arabic {subsection}}
  %\setcounter{page}{0}
}  及 \makecover 两行。
% !Mode:: "TeX:UTF-8"
%开题/中期等报告用的封面页
\hitsetup{
  %******************************
  % 注意:
  %   1. 配置里面不要出现空行
  %   2. 不需要的配置信息可以删除
  %******************************
  %
  %=====
  % 秘级
  %=====
  %
  %=========
  % 中文信息
  %=========
  ctitleone={局部多孔质气体静压局部多孔质气体静压}, %论文题目,第一行此处最多12个字,多出的字放在\ctitletwo
  ctitletwo={轴承关键技术的研究}, %选填项,论文题目第二行,第一行放不下的文字放在这里
  csubject={机械制造及其自动化},
  cclassid={1604104}, %班号,威海校区特有,仅选择威海校区时有效
  caffil={机电工程学院},
  cauthor={于冬梅},
  csupervisor={某某某教授},
  % ccosupervisor={某某某教授}, % 联合指导老师
  % 日期自动使用当前时间,若需指定按如下方式修改:
  cdate={超新星纪元},
  cstudentid={9527},
  %cstudenttype={同等学力人员}, %非全日制教育申请学位者
  %(同等学力人员)、(工程硕士)、(工商管理硕士)、
  %(高级管理人员工商管理硕士)、(公共管理硕士)、(中职教师)、(高校教师)等
  %
  %
  %配置
  %\renewcommand\thesection{\arabic {section}.}
  %\renewcommand\thesubsection{\thesection \arabic {subsection}}
  %\setcounter{page}{0}
} % 开题/中期封面
%\input{front/cover} % 毕业论文封面
\makecover
%\input{front/denotation}%物理量名称表,符合规范为主,有要求添加
%\cleardoublepage  %自定义在什么位置进行右翻页
\tableofcontents    % 中文目录,有些专业研究生开题要求添加目录,中期、本科不要求,以教学秘书发送模板要求为准, 自行添加
%\cleardoublepage  %自定义在什么位置进行右翻页
%\tableofengcontents % 英文目录,硕本不要求,开题/中期不要求

\mainmatter
%\linenumbers %debug 选项
%\layout %debug 选项
%\floatdiagram %debug 选项
%\begin{figure} %debug 选项
%\currentfloat %debug 选项
%\tryintextsep{\intextsep} %debug 选项
%\trytopfigrule{0.5pt} %debug 选项
%\trybotfigrule{1pt} %debug 选项
%\setlayoutscale{0.9} %debug 选项
%\floatdesign %debug 选项
%\caption{Float layout with rules}\label{fig:fludf} %debug 选项
%\end{figure} %debug 选项
\include{body/introduction}


\backmatter
%硕博书序
%\include{back/conclusion}   % 结论
\bibliographystyle{hithesis} %如果没有参考文献时候
%\bibliography{reference}	%开题/中期报告如果需要添加参考文献,需要注释此行,然后将此行内容添加到body文档正文末尾,否则会另起新页;对于正常毕业论文,则保留此行,去掉body文档末尾该项。
%\begin{appendix}%附录
%\input{back/appA.tex}
%\end{appendix}
%\include{back/publications}    % 所发文章
%\include{back/ceindex}    % 索引, 根据自己的情况添加或者不添加,选择自动添加或者手工添加。
%\authorization %授权
%%\authorization[saomiao.pdf] %添加扫描页的命令,与上互斥
%\include{back/acknowledgements} %致谢
%\include{back/resume}          % 博士学位论文有个人简介

%本科书序为:
%\include{body/conclusion}   % 结论
%\bibliographystyle{hithesis}
%\bibliography{reference}
%\authorization %授权
%%\authorization[saomiao.pdf] %添加扫描页的命令,与上互斥
%\include{body/acknowledgements} %致谢
%\begin{appendix}%附录
%\input{body/appendix01}%本科生翻译论文
%\end{appendix}

%开题/中期书序:


\end{document}
