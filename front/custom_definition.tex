% !Mode:: "TeX:UTF-8"
%在下面添加自定义宏包、命令

%------------以下是自定义添加的宏包(下面为示例,可删除)------------------
\usepackage{leftidx} %左上标
\usepackage{multirow} %表格多行合并
\usepackage{slashed}	% 场论slash
\usepackage{cancel}
\usepackage{setspace} %用于调整多行公式间距, \begin{spacing}{1.6} \begin{equation}..\end{equation}\end{spacing}  !!!caution:使用宏包同时也会改变表格的行距需在每个表格\begin{table}后添加:\renewcommand\arraystretch{1.55}






%------------以下是自定义添加的命令(下面为示例,可删除)------------------

%重新定义section和subsection标号格式,去掉Chapter继承标号,因为开题等报告没有Chapter层级
%\renewcommand\thesection{\arabic {section}.}
%\renewcommand\thesubsection{\thesection \arabic {subsection}}

%-------罗马数字定义-------
\makeatletter
\newcommand{\rmnum}[1]{\romannumeral #1}
\newcommand{\Rmnum}[1]{\expandafter\@slowromancap\romannumeral #1@}
\makeatother


%---------正体字母定义------------
\newcommand{\ud}{\mathrm{d}}
\newcommand{\ue}{\mathrm{e}}
\newcommand{\ui}{\mathrm{i}}

%---------度°符号定义-------------
\def\degree{\ensuremath{{}^{\circ}}} 

%---------tabular行距------------
%\renewcommand{\arraystretch}{1.25}

